\chapter{System Components}
\label{secsyscomp}
This section presents the components of the editor.

\section{The Language}
\label{subseclang}
Since {\LEFTY} is interactive, the language is designed to allow for fast
parsing and execution. The language was inspired by {\it EZ} \cite{drh-ez}.
Appendix~\ref{appgrammar} specifies the language in detail.

The language supports {\it scalars} and {\it tables}. A scalar is a number or a
character string of arbitrary length. A table is a one-dimensional array
indexed by numbers or strings.

For example, {\tt objarray} in Figure~\ref{progboxdsinitial} is a two-entry
table indexed by {\tt 0} and {\tt 1}. Each of these entries is a table with
entries for the center and the size of each box and an {\tt id} for each box.

Variables are either global, i.e., part of a global name table, or local to a
function. Expressions may or may not return a value. For example, {\tt a + b}
does not return a value when either {\tt a} or {\tt b} are not defined.

The smallest program unit is the expression. User actions on the WYSIWYG view
result in the execution of expressions. User-typed text in the program view is
a sequence of expressions. Each user action results in the immediate evaluation
of an expression. For example, if the user enters \verb+num = sqrt (4);+ in the
program view, {\tt sqrt} is called and its return value, {\tt 2}, is assigned
to {\tt num}. Once executed, the input is discarded; the only change in the
program's state is that it now contains {\tt num}. To specify code that is
meant to be executed later, the user must define a function, e.g.,

\vbox{
\begin{verbatim}
afunction = function (n) {
    num = sqrt (n);
};
\end{verbatim}
}

\noindent
``Executing'' a function declaration adds the name of the function to the
global name table. Calling {\tt afunction} assigns a value to {\tt num}, e.g.,
\verb+afunction (4);+ assigns {\tt 2} to {\tt num}.

Assignment is done either by value (scalars), or by reference (tables). For
example, after the sequence \verb+a = 1; b = a;+, {\tt a} and {\tt b} point to
two different values, while the sequence \verb+a = [ ]; b = a;+ results in both
{\tt a} and {\tt b} pointing to the same table. Functions are stored and
treated like scalars.

\section{The Program View}
The program view is a textual representation of the picture state. It displays
the name and value of each global object.

The textual representation can be long, so the editor presents an abbreviated
view by default: each name, value pair is displayed on a line.
Figure~\ref{figpviews}a shows a few of the entries in the program view for the
picture in Figure~\ref{figboxinitial}. Only the value for {\tt objnum} is
displayed, as it can fit in a single line. Other variables have an abstract
representation, which indicates whether they are functions or tables.

\begin{figure}[htb]
\begin{verbatim}
'delete' = function (...) { ... };    'delete' = function (...) { ... };
'drawbox' = function (...) { ... };   'drawbox' = function (...) { ... };
'leftdown' = function (...) { ... };  'leftdown' = function (...) { ... };
'leftmove' = function (...) { ... };  'leftmove' = function (...) { ... };
'leftup' = function (...) { ... };    'leftup' = function (...) { ... };
'move' = function (...) { ... };      'move' = function (...) { ... };
'new' = function (...) { ... };       'new' = function (...) { ... };
'objarray' = [ ... ];                 'objarray' = [
'objnum' = 2;                             0 = [ ... ];
'redraw' = function (...) { ... };        1 = [ ... ];
'reshape' = function (...) { ... };   ];
                                      'objnum' = 2;
                                      'redraw' = function (...) { ... };
                                      'reshape' = function (...) { ... };
\end{verbatim}

\hfil(a) All entries closed\hfil\hfil(b) Opening entry {\tt objarray}\hfil

\begin{verbatim}
'delete' = function (...) { ... };    'delete' = function (...) { ... };
'drawbox' = function (...) { ... };   'drawbox' = function (...) { ... };
'leftdown' = function (...) { ... };  'leftdown' = function (...) { ... };
'leftmove' = function (...) { ... };  'leftmove' = function (...) { ... };
'leftup' = function (...) { ... };    'leftup' = function (...) { ... };
'move' = function (...) { ... };      'move' = function (...) { ... };
'new' = function (...) { ... };       'new' = function (...) { ... };
'objarray' = [                        'objarray' = [
    0 = [                                 0 = [
        'id' = 0;                             'id' = 0;
        'rect' = [ ... ];                     'rect' = [ ... ];
    ];                                    ];
    1 = [ ... ];                          1 = [ ... ];
];                                    ];
'objnum' = 2;                         'objnum' = 2;
'redraw' = function (...) { ... };    'redraw' = function (...) { ... };
'reshape' = function (...) { ... };   'reshape' = function (obj, rect) {
                                          obj.rect = rect;
                                          return obj;
                                      };
                                      'zarray' = objarray;
\end{verbatim}

\hfil(c) Opening entry {\tt objarray[0]}\hfil\hfil(d) Opening entry {\tt reshape}\hfil\hfil

\caption{Various levels of abstraction on the program view}
\label{figpviews}
\end{figure}

For a more detailed view of an object, the user clicks on the line describing
the object. For example, clicking on the line for {\tt objarray} causes the
editor to expand it, as shown in Figure~\ref{figpviews}b, to show that {\tt
objarray} has two entries indexed by {\tt 0} and {\tt 1}. Clicking on the {\tt
0} entry of {\tt objarray} causes the editor to expand that entry as shown in
Figure~\ref{figpviews}c. Entry {\tt 1} remains the same, but entry {\tt 0} is
expanded. Function entries behave similarly: clicking on a function displays
the function's body. Clicking on {\tt reshape}, for example, results in the
display shown in Figure~\ref{figpviews}d.

Clicking on an expanded entry replaces that entry with its abstracted version.

If an entry points to the same value as another entry, the second entry is
shown differently. Rather than showing the same value twice, the editor shows
the duplication. For example, if we execute \verb+zarray=objarray;+,
\verb+zarray+ will be shown as in Figure~\ref{figpviews}d. This display
semantic makes it clear how much unique information is available.

Unlike in the WYSIWYG view, where changes are controlled by the program that
describes the picture, the user can do anything in the program view, including
getting the program into an inconsistent state. All the functions and tables
are visible and can be edited. This flexibility is necessary, since a
conceptual change to the program or the data usually requires a sequence of
modifications to the text of the program. Although the sequence of
modifications leaves the editor in a consistent state, individual modifications
can put the editor in an inconsistent state temporarily. For example, the user
can add a box to Figure~\ref{figboxinitial} by typing in the sequence of
commands executed by function {\tt new} in Figure~\ref{progboxedit}. After the
user has typed in the assignment for \verb+objarray[objnum]+, the program is
inconsistent: {\tt objarray} has three entries, but the value for {\tt objnum}
is still {\tt 2}. The program becomes consistent after the user types in the
command to increment {\tt objnum}.

\section{The WYSIWYG View}
The WYSIWYG view is the graphical representation of the picture. The program
that describes a picture controls the WYSIWYG view; all the objects are drawn
by the program, and all user actions are handled by the program. The WYSIWYG
view can consist of one or more widgets, such as drawing areas, buttons, lists,
text areas, and scrollable widgets.

Widgets can be manipulated using built-in functions. When a new widget is
created, {\LEFTY} adds an entry to a global table called {\tt widgets}. Each of
these entries is a table that can be used to customize the behavior of the
widget. When the user generates an event, e.g. clicks a mouse button, {\LEFTY}
searches the corresponding entry in {\tt widgets} for the appropriate callback
function. If the function cannot be found in that table, {\LEFTY} then searches
for it in the global namespace.

The most interesting type of widget in {\LEFTY} is the drawing area. Drawing
inside such a widget is handled by a set of built-in functions. The supported
graphical primitives are lines, polygons, splinegons, elliptic arcs, and text.
Each drawing area maintains its own graphics state. The built-in functions and
state variables are described in Section~\ref{secbuiltins}

When an event occurs inside a drawing area, for example, a mouse button is
pressed or released, the editor checks if a function corresponding to this
event exists. The possibilities are:

\vbox{
\begin{verbatim}
leftdown      leftmove      leftup
middledown    middlemove    middleup
rightdown     rightmove     rightup
keydown                     keyup
\end{verbatim}
}

\noindent
There is no restriction on what these functions do. The programmer must define
them as appropriate for the current picture. {\LEFTY} searches for these
functions first in the drawing area's entry in {\tt widgets}, then in the
global namespace. If a function is found, it is called with a single argument.
This argument is a table that contains information about the event. It has the
following fields.

\begin{list}{}{\renewcommand{\leftmargin}{50pt}\renewcommand{\labelwidth}{40pt}\renewcommand{\makelabel}[1]{#1\hfil}}
\item[\tt obj]
the object that the user selected with this event, or {\tt null} if no such
object could be found
\item[\tt pos]
a table with two entries, {\tt x} and {\tt y}, that hold the mouse coordinates
at the time of this event
\item[\tt pobj]
(only for {\tt move} and {\tt up} events) the object selected by the preceding
{\tt down} event.
\item[\tt ppos]
(only for {\tt move} and {\tt up} events) the mouse coordinates at the time of
the preceding {\tt down} event
\item[\tt widget]
the widget id of the drawing area where this event occurred
\item[\tt key]
(only for {\tt key} events) the ascii character of the key
\end{list}

Determining the selected object at a button press or release has two phases.
The editor determines if the mouse coordinates select a graphical primitive.
Closed shapes, for example, boxes and ellipses, are selected if the mouse
coordinates lie inside the shape. If such a primitive can be found, the editor
finds the {\LEFTY} data object associated with it.

Finding the selected graphical primitive is straightforward. The editor
maintains a data structure of all the graphical primitives in the WYSIWYG view.
When an event is received, the coordinates are used to search through this data
structure for the selected primitive. The only complication is when two or more
primitives overlap. In the box example in Section~\ref{secoverview}, boxes
could overlap. The editor does not resolve these kinds of ambiguities. One
solution would be to allow the user to rotate through all the objects that
could potentially be selected. The editor does provide a way to resolve
ambiguities created by design, such as when an object is drawn using more than
one graphical primitives. In the tree figure in Section~\ref{secintro}, a tree
node is drawn as a rectangle enclosing a label.

Finding the data object that corresponds to the selected primitive is slightly
more complex. The data object must be specified as an argument to the rendering
primitive. All rendering functions take as their second argument the object to
associate with the primitive they draw. This argument can also be {\tt null},
which effectively makes the primitive unselectable. For the tree example, the
text label of a node could be associated with {\tt null} and that would leave
the node's box as the only selectable primitive occupying that area of the
display. The box would have to be associated with the table that represents the
corresponding node of the tree. The mapping between objects and graphical
primitives is manipulated with two functions:

\vbox{
\begin{tabbing}
{\tt clearpick ({\it canvas}, {\it object})}\\
{\tt setpick ({\it canvas}, {\it object}, {\it rectangle})}\\
\end{tabbing}
\vspace{-16pt}
}

\noindent
{\tt clearpick} removes {\it object} from the mapping, and {\tt setpick}
associates the rectangular area {\it rectangle} with {\it object}. Finally,
clearing the WYSIWYG view clears the mapping.

When a drawing area is resized, or when its window is redrawn, {\LEFTY}
searches for a function called {\tt redraw}. If such a function is found in
{\tt widgets} or in the global namespace, it is called with a table as an
argument. This table contains an entry {\tt widget} which is the widget id
of the drawing area.

The label widget can display a piece of text in its rectangular area.
It provides a subset of the user-interface functions provided by the
drawing area; it provides all the {\tt up} and {\tt down} functions
mentioned above, but does not provide the {\tt move} functions. The table
passed as argument to these functions contains the {\tt widget} entry
and---for the {\tt keyup} and {\tt keydown} functions---the {\tt key}
entry.

The button widget provides a single function, {\tt pressed}. It is called
when the user clicks on the button. If such a function is found it is called
with just one entry, {\tt widget}.

When the user presses {\tt CR} in an input text widget, {\LEFTY} looks for a
function called {\tt oneline}. This function, if found, is called with two
arguments, {\tt widget} and {\tt text}. {\tt text} contains the line of text
that the user just entered.

For array widgets, {\LEFTY} tries to call a function named {\tt resize}
whenever their size changes (either through user actions, or program control).
This function is called with two arguments, {\tt widget} and {\tt size}.  {\tt
size} is an (x,y) table containing the new size of the widget. This function is
expected to return an array containing new sizes for all of the widget's
children. This array must be indexed by the widget ids of the children, and
each element must be an (x,y) table containing the size of a child.

{\LEFTY} can monitor open file descriptors. Built-in function {\tt
monitor} takes as an argument the id of an input channel (generated by {\tt
openio}), and adds it to the list of file descriptors being monitored. When
there is something to read from that file descriptor, {\LEFTY} searches the
global namespace for a function called {\tt monitorfile}. If such a function is
found, it is called with a table as an argument. This table contains an entry
{\tt fd} which is the file descriptor that is ready for reading.

Finally, when there are no X or file I/O events to handle, {\LEFTY} can
optionally execute a function called {\tt idle}. This feature can be turned
on or off using the {\tt idlerun} built-in.

\section{Inter-process Communication}
{\LEFTY} provides built-ins for communicating with other processes. This
capability can be used in several ways.

\begin{list}{}{}
\item[]
Purely for output. A process can use the editor to display some data
structures; the process does not need any code for graphical layout.
\item[]
For both input and output. The editor can be used to specify the input and to
display the result of some processing of that input. Debugging is an example;
instead of printing data structures as text or writing code to draw them, the
process being debugged simply connects to the editing server and sends the data
structures to the server for display.
\item[]
As an extension to the editor itself. There are tools displaying trees
\cite{tamassia-diagrams}, DAGs \cite{dag-spe}, delaunay triangulations
\cite{guibas}, and VLSI layouts \cite{magic}. These tools are usually large
software packages, and duplicating their functionality in the editor is a major
undertaking. Instead, the editor communicates with these tools as separate
processes. Whenever some aspect of a layout needs to be updated, the editor
sends a message asking for instructions on how to perform the update to the
appropriate process.
\end{list}

{\LEFTY} communicates with other processes by exchanging ASCII strings. This
allows {\LEFTY} to communicate with many existing tools, without having to
modify these tools at all. For example, the layout for the tree in
Figure~\ref{fig2tp}b is generated by a call to {\tt complayout}. This function
does all the calculations. If the layout were to be generated by a separate
process this function could be rewritten as follows.

\vbox{
\begin{verbatim}
complayout = function () {
    ...
    writeline (treefd, 'compute layout');
    while ((s = readline (treefd)) ~= '') {
        t = split (s, ' ');
        nodearray[ston (t[0])] = ['x' = ston (t[0]); 'y' = ston (t[0]);];
    }
    ...
};
\end{verbatim}
\vspace{-12pt}
}

\noindent
{\tt complayout} sends a message to the external process, using {\tt
writeline}, requesting a new layout. The {\tt while} loop reads back the
response from the process. Each line would consist of 3 numbers: the id of a
node and its x and y coordinates. The process must send an empty line at the
end of the transmission. {\tt treefd} is the file descriptor for communicating
with the other process.

The response from a process can also be a {\LEFTY} expression. The {\tt while}
loop above could be replaced with the following loop.

\vbox{
\begin{verbatim}
complayout = function () {
    ...
    while ((s = readline (treefd)) ~= '')
        run (s);
    ...
};
\end{verbatim}
\vspace{-12pt}
}

\noindent
{\tt run} is a built-in. It parses and execute the {\LEFTY} expression
specified by the string {\tt s}. A sample string could be {\tt nodearray[i].p =
['x' = 10; 'y' = 20;];}.

This form of remote procedure call gives processes access to {\LEFTY} functions
and data structures and should help minimize the amount of work needed to
interface a process with {\LEFTY}.

The technique of communicating by sending programs has been used in several
other systems, most notably in window systems \cite{blit,news}.

\section{Built-in Functions}
\label{secbuiltins}

{\LEFTY} built-ins can be used to perform window system / graphics operations
and to access various system resources such as files. Built-ins differ from
functions written in {\LEFTY}'s language in that they can take a variable
number of arguments. Built-in functions that are not supposed to return a
value as part of their specification, return 1 when they succeed and nothing
when they fail. This makes it possible to check whether a built-in performed
its intended function with an {\tt if}-statement.

\vbox{
\begin{verbatim}
if (~setwidgetattr (wid, ['text' = 'some text';]))
    echo ('setwidgetattr failed');
\end{verbatim}
\vspace{-12pt}
}

\noindent
Built-ins that are supposed to return specific values, also
return nothing to indicate failure.

\subsubsection{Widget Functions}

\begin{flushleft}
\it widgetid \tt = createwidget (\it parentid, attr\tt )\\
\tt setwidgetattr (\it widgetid, attr\tt )\\
\it attr \tt = getwidgetattr (\it widgetid, keys\tt )\\
\tt destroywidget (\it widgetid\tt )\\
\end{flushleft}\vspace{-2\itemsep}
These functions are used to create, modify, and destroy widgets.  {\tt
createwidget} creates a new widget and returns its id. This id is a small
integer. {\tt createwidget} creates a new table, indexed by {\it widgetid},
under the {\tt widgets} global table. {\it parentid} is the id of the parent
widget. {\it attr} is a table of attributes, such as type, size, name, etc.
Attribute {\tt type} must be specified, but if some other attributes are not
set, default values are used instead. {\tt setwidgetattr} sets one or more
attributes for the specified widget (except for {\tt type}). {\tt
getwidgetattr} returns the current values of the attributes specified by {\it
keys}.  {\it keys} is an indexed array of attribute names. For example, if {\it
keys} is set to {\tt [0 = 'name'; 1 = 'size';]}, the returned {\it attr} table
will contain two entries, {\tt ['name' = ...; 'size' = ...;]}. {\tt
destroywidget} destroys the specified widget and any children that it might
have.

Tables~\ref{tabwidgets1} and ~\ref{tabwidgets2} show the available widgets.

\begin{table}[htb]
\begin{tabular}{|l|p{0.9in}p{1.1in}|p{2.8in}|} \hline
Type&Attributes&Attr. type&Description\\ \hline

\tt view&\tt origin\newline size\newline name\newline zorder&
table of (x, y)\newline table of (x, y)\newline string\newline string&
A top level window. It may contain exactly one child. {\tt zorder} can be used
to push / pop the view (values {\tt "top"}, {\tt "bottom"}).\\ \hline

\tt text&\tt size\newline borderwidth\newline text\newline mode\newline appendtext&
table of (x, y)\newline integer\newline string\newline string\newline string&
A widget that can display (and optionally edit) text. {\tt mode} can be one of
{\tt "oneline"}, {\tt "input"}, or {\tt "output"}. For mode {\tt line},
{\LEFTY} tries to execute the {\tt func} callback whenever \verb+CR+ is
pressed. {\tt appendtext} appends a string to the string already displayed by
the widget.\\ \hline

\tt scroll&\tt size\newline borderwidth\newline childcenter\newline mode&
table of (x, y)\newline integer\newline table of (x, y)\newline string&
A widget that can contain a---potentially larger---child widget and let the
user scroll through it. {\tt childcenter} may not be specified until the scroll
widget has a child widget. {\tt childcenter} aligns the child so that the
child's {\tt childcenter} coordinates are at the center of the scroll
widget. {\tt mode} can be set to {\tt "forcebars"} to make scrollbars appear
even when the child widget is small enough to fit inside the scroll widget.\\
\hline

\tt array&\tt size\newline borderwidth\newline mode\newline layout&
table of (x, y)\newline integer\newline string\newline string&
A widget that can take a list of children widgets and display them either as a
horizontal or a vertical list. {\tt mode} can be one of {\tt "horizontal"}, or
{\tt "vertical"}. {\tt layout} controls whether the widget rearranges its
children every time there is some change. If set to {\tt "off"} the widget will
stop rearranging its children until {\tt layout} is set to {\tt "on"} again.\\
\hline
\end{tabular}
\caption{Widget types part 1}
\label{tabwidgets1}
\end{table}

\begin{table}[htb]
\begin{tabular}{|l|p{0.9in}p{1.1in}|p{2.8in}|} \hline
Type&Attributes&Attr. type&Description\\ \hline

\tt button&\tt size\newline borderwidth\newline text&
table of (x, y)\newline integer\newline string&
A widget that can display a text label and execute the callback {\tt pressed}
when it is selected.\\ \hline

\tt canvas&\tt size\newline borderwidth\newline cursor\newline color\newline\newline viewport\newline window&
table of (x, y)\newline integer\newline string\newline array of (r, g, b)\newline and strings\newline table of (x, y)\newline 2 tables of (x, y)&
A drawing area. {\tt cursor} must be the name of a cursor bitmap,
e.g. \verb+"watch"+ or \verb+"default"+. {\tt color} is an array of RGB values
and color names. Colors {\tt 0} and {\tt 1} are predefined to be the background
and foreground colors. {\tt viewport} sets the size in pixels of the drawing
area. {\tt window} sets the mapping between drawing coordinates and pixel
coordinates. The default value for {\tt window} is (0,0) - (1,1). The origin is
at the lower left side.\\ \hline

\tt label&\tt size\newline borderwidth\newline text&
table of (x, y)\newline integer\newline string&
A widget that can display a text label and execute several callbacks depending
on the mouse or keyboard buttons used.\\ \hline

\tt ps&\tt origin\newline size\newline name\newline mode\newline color\newline window&
table of (x, y)\newline table of (x, y)\newline string\newline string\newline array of (r, g, b)\newline 2 tables of (x, y)&
A postscript file. {\tt name} is the file name. {\tt mode} can be {\tt
"landscape"}.\\ \hline
\end{tabular}
\caption{Widget types part 2}
\label{tabwidgets2}
\end{table}

\subsubsection{Graphics Functions}

\begin{flushleft}
\tt clear (\it canvasid\tt )\\
\tt clearpick (\it canvasid, object\tt )\\
\tt setpick (\it canvasid, object, rect\tt )\\
\end{flushleft}\vspace{-2\itemsep}
{\tt clear} clears the drawing area whose id is {\it canvasid} and the table
that contains the mapping between data objects and graphical objects.  {\tt
clearpick} removes {\it object} from the mapping table, and {\tt setpick} maps
the rectangular area specified by {\it rect} to {\it object}.

\begin{flushleft}
\it item = \tt displaymenu (\it widgetid, menu\tt)\\
\end{flushleft}\vspace{-2\itemsep}
{\tt displaymenu} pops up the menu specified by {\it menu} inside the widget
specified by {\it widgetid} (which must be either a canvas or a label
widget). {\it menu} must be a table of number-string pairs. When the user
selects one of the string entries {\tt displaymenu} returns the number
associated with that string. If the user dismisses the menu, {\tt -1} is
returned.

\begin{flushleft}
\it reply \tt = ask (\it prompt [, type, args]\tt)\\
\end{flushleft}\vspace{-2\itemsep}
Prompts the user for information; it displays the {\it prompt} string in a
dialog box and waits for the user to type or select a reply, which is returned
as the value of {\tt ask}. If {\it type} is the string {\tt "file"}, the dialog
box shows the contents of the directory specified in {\it args}. If
{\it type} is {\tt "choice"}, {\it args} must be a string of the form
{\tt "<choice a>|<choice b>|..."}. Each choice string appears as a button
that the user can click to select. If {\it type} is {\tt string}, the dialog
box has a text field that the user can type in. {\it args} in this case
is the initial value of the text field. Finally, if {\it type} is not
specified, {\tt "string"} is assumed.

\begin{flushleft}
\tt setgfxattr (\it canvasid, attr\tt )\\
\it attr \tt = getgfxattr (\it canvasid, keys\tt )\\
\end{flushleft}\vspace{-2\itemsep}
{\tt setgfxattr} sets attributes in the graphics state. Each drawing area has
its own state variables. {\tt setgfxattr} sets these attributes
permanently. These attributes can also be set on a per rendering call
basis. {\tt getgfxattr} returns the current values of the attributes specified
by keys. {\it keys} is an indexed array of attribute names. For example, if
{\it keys} is set to {\tt [0 = 'mode'; 1 = 'width';]}, the returned {\it attr}
table will contain two entries, {\tt ['mode' = ...; 'width' = ...;]}.

The graphics state consists of the variables shown in Table~\ref{tabgattr}.

\begin{table}[htb]
\begin{tabular}{|l|l|p{0.7in}|l|p{3in}|} \hline
Name&Type&Range&Default&Description\\ \hline
\tt color&integer&{\tt 0-255}&\tt 1&
The current drawing color.\\
\tt width&integer&{\tt >= 0}&\tt 0&
The current line width.\\
\tt mode&string&{\tt 'src'}\newline {\tt 'xor'}&\tt 'src'&
The current drawing mode.\\
\tt fill&string&{\tt 'on'}\newline {\tt 'off}&\tt 'off'&
Whether polygons and arcs should be drawn filled or outlined.\\
\tt style&string&{\tt 'solid'}\newline {\tt 'dashed'}\newline {\tt 'dotted'}&\tt 'solid'&
The current line style.\\
\hline
\end{tabular}
\caption{Graphics state}
\label{tabgattr}
\end{table}

\begin{flushleft}
\tt arrow (\it canvasid, object, p1, p2 [, attr]\/\tt )\\
\tt line (\it canvasid, object, p1, p2 [, attr]\/\tt )\\
\tt box (\it canvasid, object, rect [, attr]\/\tt )\\
\tt polygon (\it canvasid, object, pointarray [, attr]\/\tt )\\
\tt splinegon (\it canvasid, object, pointarray [, attr]\/\tt )\\
\tt arc (\it canvasid, object, center, size [, attr]\/\tt )\\
\tt text (\it canvasid, object, pos, string, fontname, fontsize, just [,attr]\/\tt )\\
\it size \tt = textsize (\it canvasid, object, fontname, fontsize\tt )\\
\end{flushleft}\vspace{-2\itemsep}
The final argument in most of these functions can be used to change the
graphics state for the execution of that function. {\tt splinegon} draws a
piecewise bezier spline curve. {\tt fontname} must be an X font name or a
postscript font name. For ISO style font names, if the name contains the
sequence {\tt \%d}, this sequence will be replaced by the appropriate font
size.  {\tt just} is a two letter string that controls the justification of the
string. The first letter may be {\tt l}, {\tt c}, or {\tt r} for left, center,
or right justified strings. The second letter specifies the vertical
justification and can be one of {\tt u}, {\tt c}, {\tt d}.

\subsubsection{Bitmap Functions}

\begin{flushleft}
\it bitmapid \tt = createbitmap (\it widgetid, size\tt )\\
\tt destroybitmap (\it bitmapid\tt )\\
\it bitmapid \tt = readbitmap (\it widgetid, fileid\tt )\\
\tt writebitmap (\it fileid, bitmapid\tt )\\
\tt bitblt (\it canvasid, object, point, origin, bitmapid, mode\tt )\\
\end{flushleft}\vspace{-2\itemsep}
These functions are used to create, modify, and destroy bitmaps.  {\tt
createbitmap} creates a new bitmap and returns its id. This id is a small
integer. {\tt createbitmap} creates a new table, indexed by {\it bitmapid},
under the {\tt bitmaps} global table. {\it widgetid} is the id of the canvas
widget associated with the bitmap. A bitmap can only be displayed in its
associated canvas. {\it size} is the size of the bitmap.
{\tt destroybitmap} destroys the specified bitmap. {\tt readbitmap}
reads a bitmap from file descriptor {\it fileid} and returns a new bitmap
id. The bitmap is assumed to be in PPM format. {\tt savebitmap} writes
the specified bitmap to file descriptor {\it fileid}. {\tt bitblt} copies
pixels between canvas {\it canvasid} and bitmap {\it bitmapid}. If {\it mode}
is {\tt 'c2b'} pixels are copied from the canvas to the bitmap. If {\it mode}
is {\tt 'b2c'} pixels are copied from the bitmap to the canvas. Pixels
are copied from the source (bitmap or canvas) starting at point {\it point}
to the destination starting at the origin of the {\it rect} rectangle.
The size of the rectangle specifies the amount of pixels to copy.

Bitmap are scaled when copied to / from canvases. For example, if the
canvas window to viewport ratio is 2.0, a bitmap drawn in the canvas will
be scaled to 0.5 of its size.

\subsubsection{Input / Output functions}

\begin{flushleft}
\it id \tt= openio (\it type, name, mode [, format]\/\tt )\\
\tt closeio (\it id [, flag]\/\tt )\\
\it string \tt = readline (\it id\tt )\\
\it string \tt = read (\it id\tt )\\
\tt writeline (\it id, string\tt )\\
\it table \tt = readgraph (\it id\tt )\\
\tt writegraph (\it id, table, flag\tt )\\
\it table \tt = parsegraphlabel (\it label, rects\tt )\\
\end{flushleft}\vspace{-2\itemsep}
These functions handle input and output for a variety of connections, such as
regular files, pipes, and sockets. {\it type} is a string whose value can be
one of {\tt 'file'}, {\tt 'pipe'}, {\tt 'socket'}, or {\tt 'cs'}. For regular
files, file {\it name} is opened for reading or writing, depending on {\it
mode}.  {\it mode} can be one of {\tt 'r'}, {\tt 'w'}, or {\tt 'w+'}. For pipes
and sockets, {\it name} is the name of an executable. If the name does not
begin with {\tt /}, or {\tt .}, the executable is searched, first in the path
defined by the environment variable {\tt LEFTYPATH}, then in {\tt
PATH}. Finally, if {\it format} is specified, it customizes the way an
executable is invoked. In {\it format}, a {\tt \%} followed by a letter
specifies a formatting directive. The following directives are currently
recognized.

\begin{list}{}{}
\item[\tt \%e]
the full path name for the executable.
\item[\tt \%i]
the input file descriptor (for pipes).
\item[\tt \%o]
the output file descriptor (for pipes).
\item[\tt \%h]
the hostname (for sockets).
\end{list}

An arbitrary shell command can be executed by calling {\tt openio} with {\it
name} set to {\tt "ksh"} (or any other shell) and {\it format} set to {\tt
concat ('\%e -c "', {\it cmd}, '"')}, where {\it cmd} is the shell command.
For sockets, {\LEFTY} creates an internet socket, starts up the executable,
then waits for the executable to connect to that socket. The executable must
try to connect to the host and port specified by {\tt \%h} and {\tt \%i}.
{\tt 'cs'} can be used
to establish a {\it libcs}-style connection. {\tt name} in this case is the
{\it libcs} name for a service. The optional {\tt flag} parameter can be set to
{\tt "kill"} to make {\LEFTY} send the kill signal to the child process (if
such exists) after it has closed the file descriptor.  {\tt readline} reads a
full line and returns it (stripping the newline character). {\tt readgraph}
reads a graph in {\DOT}'s language and returns it as a table. {\tt writegraph}
writes out the graph {\it table}. If {\it flag} is set to {\tt 1} {\LEFTY}
will attach extra attributes to edges to help identify them when this graph
is read back in. This is used when {\LEFTY} communicates with {\DOT}. {\tt
parsegraphlabel} takes a {\DOT}-style record label and the corresponding string
of coordinates for the record fields and returns a hierachical table.  Each
entry in this table contains either the text and coordinates of a field, or a
sub-table of fields.

\subsubsection{Math Functions}

\begin{flushleft}\tt
\it value \tt = atan (\it y, x\tt )\\
\it value \tt = cos (\it angle\tt )\\
\it value \tt = sin (\it angle\tt )\\
\it value \tt = sqrt (\it number\tt )\\
\it value \tt = random (\it number\tt )\\
\it integer \tt = toint (\it number\tt )
\end{flushleft}\vspace{-2\itemsep}
{\tt angle} is assumed to be in degrees. {\tt toint} truncates the decimal part
of {\it number}.

\subsubsection{Miscellaneous Functions}

\begin{flushleft}
\tt dump (\it ...\tt )\\
\tt echo (\it arg1, ...\tt )\\
\end{flushleft}\vspace{-2\itemsep}
{\tt echo} prints out each arguments by appending them one after the other in
the same line. {\tt echo} does not handle tables or functions. {\tt dump}
prints its arguments separated by newlines. It can handle any type of {\LEFTY}
object. If {\tt dump} is called with no arguments, it prints the entire
namespace.

\begin{flushleft}
\it object2 \tt = copy (\it object1\tt )\\
\tt remove (\it key [, table]\/\tt )\\
\it size \tt = tablesize (\it table\tt )\\
\it size \tt = strlen (\it string\tt )\\
\it table \tt = split (\it string, delimiter\tt )\\
\it string \tt = concat (\it arg1 [, ...]\/\tt )\\
\it string \tt = ntos (\it number\tt )\\
\it number \tt = ston (\it string\tt )\\
\it string \tt = quote (\it scalar [, qset [, qchar]]\/\tt )\\
\end{flushleft}\vspace{-2\itemsep}
{\tt copy} makes a complete copy of {\it object1}. This is useful for assigning
tables by value. {\tt remove} removes {\it key}, either from {\it table}, or
from the global namespace. {\tt tablesize} returns the number of entries in a
table. {\tt strlen} returns the number of characters in {\it string}.  {\tt
split} splits {\it string} in words and returns a table (indexed from 0 and
up), where each entry is a word. {\it delimiter} is a one character string that
is used to break {\it string} into words. Each occurance of {\it delimiter}
separates two words. The only exception is when {\it delimiter} is the space
character; all leading and trailing spaces are ignored and multiple spaces are
treated as a single space. {\tt concat} concatenates all its arguments into one
string. {\tt ntos} converts a number to a string. {\tt ston} converts a
string to a number. {\tt quote} returns a string representation of the input
{\it scalar}. When a character in {\it scalar} is in the {\it qset} string it
is escaped by prepending the backquote character. If {\it qset} is
not specified, the default \verb+'"+ is used. If {\it qchar} is specified,
a {\it qchar} is added at the beginning and the end of the output string.

\begin{flushleft}
\tt load (\it string\tt )\\
\tt run (\it string\tt )\\
\tt exit ()\\
\end{flushleft}\vspace{-2\itemsep}
{\tt load} parses and executes {\LEFTY} statements from the file specified by
{\it string}. If the file does not start with {\tt /} or {\tt .}, it is
searched in the path specified by the environment variable {\tt LEFTYPATH}.
{\tt run} parses and executes the {\LEFTY} statements in {\it string}.  {\tt
exit} quits the editor.

\begin{flushleft}
\tt txtview (\it mode\tt )\\
\end{flushleft}\vspace{-2\itemsep}
{\tt txtview} turns the program view on or off. {\it mode} can be one of {\tt
'on'} or {\tt 'off'}.

\begin{flushleft}
\tt monitor (\it fileid, mode\tt )\\
\end{flushleft}\vspace{-2\itemsep}
{\tt monitor} turns on or off the monitoring of a file for input. {\it mode}
can be one of {\tt 'on'} or {\tt 'off'}. {\it fileid} is an id returned from
{\tt openio}. When a file descriptor becomes ready for reading, {\LEFTY} calls
the {\tt monitorfile} callback.

\begin{flushleft}
\tt idlerun (\it mode\tt )\\
\end{flushleft}\vspace{-2\itemsep}
{\tt idlerun} can be used to control what {\LEFTY} does when there are no
events to handle. {\it mode}
can be one of {\tt 'on'} or {\tt 'off'}. The default mode if {\tt 'off'}.
Setting {\tt mode} to {\tt 'on'}, instructs {\LEFTY} to keep running the
{\tt idle} callback unless there are X or file events to handle.
Setting {\tt mode} to {\tt 'off'}, instructs {\LEFTY} to just block waiting
for events to handle.

\begin{flushleft}
\tt sleep (\it useconds\tt )\\
\it useconds \tt = time ()\\
\end{flushleft}\vspace{-2\itemsep}
{\tt sleep} pauses execution for {\it useconds} microseconds.
{\tt time} returns the time of day in microseconds.

\begin{flushleft}
\tt system (\it string\tt )\\
\end{flushleft}\vspace{-2\itemsep}
{\tt system} executes {\it string} as a shell command. It waits until the
command finishes.

\begin{flushleft}
\it value \tt = getenv (\it name\tt )\\
\tt putenv (\it name, value\tt )\\
\end{flushleft}\vspace{-2\itemsep}
{\tt getenv} returns the value associated with the environment variable
{\it name}. {\tt putenv} sets the value for the environment variable
{\it name} to value {\it value}. Appendix~\ref{apprunning} describes the
{\LEFTY}-specific environment variables.
