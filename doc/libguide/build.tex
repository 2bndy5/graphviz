\section{Compiling and linking}
\label{sec:build}
All of the necessary include files and libraries are available
in the {\tt include} and {\tt lib} directories where \gviz\
is installed. At the simplest level, all an application needs
to do to use the layout algorithms is to include {\tt gvc.h},
which provides (indirectly) all of the \gviz\ types and functions,
compile the code,
and link the program with the necessary libraries.

For linking, the application should use the \gviz\ libraries 
\begin{itemize}
\item {\tt gvc}
\item {\tt common}
\item {\tt dotgen}
\item {\tt neatogen}
\item {\tt fdpgen}
\item {\tt twopigen}
\item {\tt circogen}
\item {\tt pack}
\item {\tt vpsc}\footnote{
Only if \gviz\ was built with IPSEPCOLA defined.}
\item {\tt plugin}
\item {\tt pathplan}
\item {\tt gd}\footnote{
The {\tt gd} library provided by \gviz\ is essentially
the same as the standard version. Whether or not the \gviz\ version is built
depends on how the software is configured at build time. Also, the \gviz\
version may be named {\tt gvgd}.
}
\item {\tt graph}
\item {\tt cdt}
\end{itemize}
On non-Windows sytems, the first 9, from {\tt gvc} to {\tt vpsc},
are combined into a single {\tt gvc} library. In addition, if the
system was compiled to use plug-ins, the linker need to reference
the plug-in libraries, as these will be dynamically loaded on demand. 

In addition, the following additional libraries may be necessary, depending
on the options set when \gviz\ was built.
\begin{itemize}
\item {\tt expat}
\item {\tt fontconfig}
\item {\tt freetype2}
\item {\tt jpeg}
\item {\tt png}
\item {\tt z}
\item {\tt ltdl}
\end{itemize}
For Windows, the \gviz\ binary package provides these latter libraries as
{\tt ft.lib libexpat.lib libexpatw.lib jpeg.lib png.lib z.lib}.\footnote{
At present, the Windows version of \gviz\ does not support {\tt fontconfig}
or {\tt ltdl}.}
Also, in some environments, it may be necessary to link
in the standard C math library.

If \gviz\ is built and installed with the GNU build tools, 
there is a {\tt pkg-config} program created in the {\tt bin} 
directory which can be used
to obtain the include file and library information for 
a given installation.
If GNU {\tt make} is used, a sample {\tt Makefile} for building the
programs listed in Appendices~\ref{sec:simple}, \ref{sec:dot} 
and \ref{sec:demo}\footnote{They
can also be found, along with the {\tt Makefile}, in the
{\tt dot.demo} directory of the \gviz\ source.}
could have the form:

\begin{verbatim}
CFLAGS=`pkg-config libgvc --cflags` -Wall -g -O2
LDFLAGS=-Wl,--rpath -Wl,`pkg-config libgvc --variable=libdir` `pkg-config libgvc --libs`

all: simple dot demo

simple: simple.o
dot: dot.o
demo: demo.o

clean:
    rm -rf simple dot demo *.o
\end{verbatim}

