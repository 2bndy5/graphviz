\section{Graphics renderers}
\label{sec:renderers}

All graph output done in \gviz\ goes through a renderer
with the type {\tt gvrender\_engine\_t}, used in the call
to {\tt gvRender}.
In addition to the renderers which are part of the library, an application
can provide its own, allowing it to specialize or control the output
as necessary. See Section~\ref{sec:plugin} for further details.

As in the layout phase invoked by {\tt gvLayout}, all control over
aspects of rendering are handled via graph attributes. For example,
the attribute {\tt outputorder} determines whether all edges are drawn
before any nodes, or all nodes are drawn before any edges.

Before describing the renderer functions in detail, it may be helpful
to give an overview of how output is done. Output can be viewed as a hierarchy
of document components. At the highest level is the job, representing 
an output format and target. Bound to a job might be multiple graphs,
each embedded in some universal space. Each graph may be partitioned
into multiple layers as determined by a graph's {\tt layers} attribute,
if any. Each layer may be divided into a 2-dimensional array
of pages. A page will then contain nodes, edges, and clusters. Each of
these may contain an HTML anchor.
During rendering, each component
is reflected in paired calls to its corresponding {\tt begin\_...} and
{\tt end\_...} functions. The layer and anchor components are omitted if
there is only a single layer or the enclosing component has no browser
information. 

Figure~\ref{fig:codegen} lists the names and type signatures of
the fields of {\tt gv\_render\_engine\_t}, which are used to emit the components
described above.\footnote{Any types mentioned in this section are
either described in this section or in Appendix~\ref{sec:types}.}
All of the functions take a {\tt GVJ\_t*} value, which
contains various information about the current rendering, such as the
output stream, if any, or the device size and resolution.
Section~\ref{sec:gvjfields} describes this data structure.

Most of the functions handle the nested graph structure.
All graphics output is handled by the {\tt textpara}, {\tt ellipse},
{\tt polygon}, {\tt beziercurve}, and {\tt polyline} functions.
The relevant drawing information such as color and pen style
is available through the {\tt obj} field of the
{\tt GVJ\_t*} parameter. 
This is described in Section~\ref{sec:obj}.
Font information is passed with the text.

We note that, in \gviz, each node, edge or cluster in a graph 
has a unique {\tt id}
field, which can be used as a key for storing and accessing the object.

\begin{figure*}[htb]
\centering
\begin{tabular}{|l|} \hline
void (*begin\_job) (GVJ\_t*);  \\
void (*end\_job) (GVJ\_t*); \\
void (*begin\_graph) (GVJ\_t*); \\
void (*end\_graph) (GVJ\_t*); \\
void (*begin\_layer) (GVJ\_t*, char*, int, int); \\
void (*end\_layer) (GVJ\_t*); \\
void (*begin\_page) (GVJ\_t*); \\
void (*end\_page) (GVJ\_t*); \\
void (*begin\_cluster) (GVJ\_t*, char*, long); \\
void (*end\_cluster) (GVJ\_t*); \\
void (*begin\_nodes) (GVJ\_t*); \\
void (*end\_nodes) (GVJ\_t*); \\
void (*begin\_edges) (GVJ\_t*); \\
void (*end\_edges) (GVJ\_t*); \\
void (*begin\_node) (GVJ\_t*, char*, long); \\
void (*end\_node) (GVJ\_t*); \\
void (*begin\_edge) (GVJ\_t*, char*, bool, char*, long); \\
void (*end\_edge) (GVJ\_t*); \\
void (*begin\_anchor) (GVJ\_t*, char*, char*, char*); \\
void (*end\_anchor) (GVJ\_t*); \\
void (*textpara) (GVJ\_t*, pointf, textpara\_t*); \\
void (*resolve\_color) (GVJ\_t*, gvcolor\_t*); \\
void (*ellipse) (GVJ\_t*, pointf*, int); \\
void (*polygon) (GVJ\_t*, pointf*, int, int); \\
void (*beziercurve) (GVJ\_t*, pointf*, int, int, int, int); \\
void (*polyline) (GVJ\_t*, pointf*, int); \\
void (*comment) (GVJ\_t*, char*); \\ \hline
\end{tabular}
\caption{Interface for a renderer}
\label{fig:codegen}
\end{figure*}
In the following, we describe the functions in more detail, though
many are self-explanatory. 
All positions and sizes are in points.

\begin{description}
\item[{\tt begin\_job(job)}]
Called at the beginning of all graphics output for a graph,
which may entail drawing multiple layers and multiple pages.
\item[{\tt end\_job(job)}]
Called at the end of all graphics output for graph. The output stream
is still open, so the renderer can append any final information
to the output.
\item[{\tt begin\_graph(job)}]
Called at the beginning of drawing a graph.
The actual graph is available as {\tt job->obj->u.g}.
\item[{\tt end\_graph(job)}]
Called when the drawing of a graph is complete.
\item[{\tt begin\_layer(job,layerName,n,nLayers)}]
Called at the beginning of each layer, only if ${\tt nLayers} > 0$.
The {\tt layerName} parameter is the logical layer name given in
the {\tt layers} attribute. The layer has index {\tt n} out of
{\tt nLayers}, starting from 0.
\item[{\tt end\_layer(job)}] 
Called at the end of drawing the current layer.
\item[{\tt begin\_page(job)}] 
Called at the beginning of a new output page.
A page will contain a rectangular portion of the drawing of the graph.
The value {\tt job->pageOffset} gives the lower left corner of the
rectangle in layout coordinates.
The point {\tt job->pagesArrayElem} is the index of the page in the array of 
pages, with the page in the lower left corner indexed by (0,0). 
The value {\tt job->zoom} provides a scale factor by which the drawing should
be scaled. The value {\tt job->rotation}, if non-zero, indicates that the
output should be rotated by $90^{\circ}$ counterclockwise.
\item[{\tt end\_page(job)}] 
Called when the drawing of a current page is complete.
\item[{\tt begin\_cluster(job)}]
Called at the beginning of drawing a cluster subgraph.
The actual cluster is available as {\tt job->obj->u.sg}.
\item[{\tt end\_cluster(job)}]
Called at the end of drawing the current cluster subgraph.
\item[{\tt begin\_nodes(job)}]
Called at the beginning of drawing the nodes on the current page.
Only called if the graph attribute {\tt outputorder} was set to a non-default
value.
\item[{\tt end\_nodes(job)}]
Called when all nodes on a page have been drawn.
Only called if the graph attribute {\tt outputorder} was set to a non-default
value.
\item[{\tt begin\_edges(job)}]
Called at the beginning of drawing the edges on the current page.
Only called if the graph attribute {\tt outputorder} was set to a non-default
value.
\item[{\tt end\_edges()}]
Called when all edges on the current page are drawn.
Only called if the graph attribute {\tt outputorder} was set to a non-default
value.
\item[{\tt begin\_node(job)}]
Called at the start of drawing a node.
The actual node is available as {\tt job->obj->u.n}.
\item[{\tt end\_node(job)}]
Called at the end of drawing the current node.
\item[{\tt begin\_edge(job)}]
Called at the start of drawing an edge.
The actual edge is available as {\tt job->obj->u.e}.
\item[{\tt end\_edge(job)}]
Called at the end of drawing the current edge.
\item[{\tt begin\_anchor(job,href,tooltip,target)}]
Called at the start of an anchor context associated
with the current node, edge, or graph, or its label, assuming the graph
object or its label has a {\tt URL} or {\tt href} attribute. The
{\tt href} parameter gives the associated href, while
{\tt tooltip} and {\tt target} supply any tooltip or target information.
If the object has no tooltip, its label will be used. 
If the object has no target attribute, this parameter will be {\tt NULL}.

If the anchor information is attached to a graph object,
the {\tt begin\_anchor} and {\tt end\_anchor} calls 
enclose the {\tt begin\_...} and {\tt end\_...} calls on the object.
If the anchor information is attached to part of an object's label,
the {\tt begin\_anchor} and {\tt end\_anchor} calls 
enclose the rendering of that part of the label plus any subparts.
\item[{\tt end\_anchor(job)}]
Called at the end of the current anchor context.
\item[{\tt textpara(job, p, txt)}]
Draw text at point {\tt p} using the specified font and fontsize and color. 
The {\tt txt} argument provides the text string {\tt txt->str},
stored in UTF-8,
a calculated width of the string {\tt txt->width} and 
the horizontal alignment {\tt txt->just} of the string in relation to {\tt p}.
The values {\tt txt->fontname} and {\tt txt->fontname} give the desired
font name and font size, the latter in points.

The base line of the text is given by {\tt p.y}. The interpretation
of {\tt p.x} depends upon the value of {\tt txt->just}. Basically,
{\tt p.x} provides the anchor point for the alignment.
\begin{center}
\begin{tabular}{ll}
{\tt txt->just} & {\tt p.x} \\ \hline
{\tt 'n'} & Center of text \\
{\tt 'l'} & Left edge of text \\
{\tt 'r'} & Right edge of text \\
\end{tabular}
\end{center}
The leftmost x coordinate of the text, the parameter most graphics
systems use for text placement, is given by 
{\tt p.x + {\em j} * txt->width}, where {\em j} is 0.0 (-0.5,-1.0)
if {\tt txt->just} is {\tt 'l'}({\tt 'n'},{\tt 'r'}), respectively.
This representation allows the renderer to accurately compute
the point for text placement that is appropriate for its format, as
well as use its own mechanism for computing the width of the string.
\item[{\tt resolve\_color(job, color)}]
Resolve a color. The {\tt color} parameter points to a color representation
of some particular type. The renderer can use this information to resolve
the color to a representation appropriate for it. See Section~\ref{sec:color}
for more details.
\item[{\tt ellipse(job, ps, filled)}]
Draw an ellipse with center at {\tt ps[0]}, with horizontal and vertical 
half-axes {\tt ps[1].x - ps[0].x} and {\tt ps[1].y - ps[0].y} 
using the current pen color and line style.
If {\tt filled} is non-zero, the ellipse should be filled with the current
fill color.
\item[{\tt polygon(job, A, n, filled)}]
Draw a polygon with the {\tt n} vertices given in the array {\tt A}, 
using the current pen color and line style.
If {\tt filled} is non-zero, the polygon should be filled with the current
fill color.
\item[{\tt beziercurve(job, A, n, arrow\_at\_start, arrow\_at\_end, filled)}]
Draw a B-spline with the {\tt n} control points given in {\tt A}. This will
consist of $(n - 1)/3$ cubic Bezier curves. The spline should be drawn
using the current pen color and line style.
If the renderer has specified that it does not want to do its own arrowheads
(cf. Section~\ref{sec:plugin}),
the parameters 
{\tt arrow\_at\_start} and {\tt arrow\_at\_end}
will both be 0. Otherwise, if {\tt arrow\_at\_start} ({\tt arrow\_at\_end}) 
is true, the function should draw an arrowhead at the 
first (last) point of {\tt A}.
If {\tt filled} is non-zero, the bezier should be filled with the current
fill color.
\item[{\tt polyline(job,A,n)}]
Draw a polyline with the {\tt n} vertices given in the array {\tt A}, 
using the current pen color and line style.
\item[{\tt comment(job, text)}]
Emit text comments related to a graph object.
For nodes, calls will pass the node's name and any {\tt comment}
attribute attached to the node.
For edges, calls will pass a string description of the edge 
and any {\tt comment} attribute attached to the edge.
For graphs and clusters, a call will pass a
any {\tt comment} attribute attached to the object.
\end{description}

Although access to the graph object being drawn is available through
the {\tt GVJ\_t} value, a renderer
can often perform its role by just implementing the basic graphics operations.
It need have no information about graphs or the related \gviz\ data structures.
Indeed, a particular renderer need not define any particular rendering
function, since a given entry point will only be called if non-NULL.

\subsection{The {\tt GVJ\_t} data structure}
\label{sec:gvjfields}
We now describe some of the more important fields in the {\tt GVJ\_t} structure,
concentrating on those regarding output. There are additional fields
relevant to input and GUIs.
\begin{description}
\item[{\tt common}]
This points to various information valid throughout the duration
of the application using \gviz. In particular, {\tt common->user}
gives the user name associated to the related {\tt GVC\_t} value
(see Section~\ref{sec:gvc}), and {\tt common->info} contains \gviz\ version
information, as described in Section~\ref{sec:info}.
\item[{\tt output\_file}]
The {\tt FILE*} value for an open stream on
which the output should be written, if relevant.
\item[{\tt pagesArraySize}]
The size of the array of pages in which the graph will be output,
given as a {\tt point}.
If {\tt pagesArraySize.x} or {\tt pagesArraySize.y} is greater than one, 
this indicates that a
page size was set and the graph drawing is too large to be printed on a
single page.
Page (0,0) is the page containing the bottom, lefthand corner
of the graph drawing; page (1,0) will contain that part of the graph
drawing to the right of page (0,0); etc. 
\item[{\tt bb}]
The bounding box of the layout in the universal space
in points. It has type {\tt boxf}.
\item[{\tt boundingBox}]
The bounding box of the layout in the device space in
device coordinates. It has type {\tt box}.
\item[{\tt layerNum}]
The current layer number.
\item[{\tt numLayers}]
The total number of layers.
\item[{\tt pagesArrayElem}]
The row and column of the current page.
\item[{\tt pageOffset}]
The origin of the current page in the universal space in points. 
\item[{\tt zoom}]
Factor by which the output should be scaled.
\item[{\tt rotation}]
Indicates whether or not the rendering should be rotated.
\item[{\tt obj}]
Information related to the current object being rendered.
This is a pointer of a value of type {\tt obj\_state\_t}.
See Section~\ref{sec:obj} for more details.
\end{description}

\subsection{Inside the {\tt obj\_state\_t} data structure}
\label{sec:obj}
A value of type {\tt obj\_state\_t} encapsulates various information
pertaining to the current object being rendered. In particular, it
provides access to the current object, and provides the style
information for any rendering operation.
Figure~\ref{fig:obj} notes some of the more useful fields in the
structure. 
\begin{figure*}[htb]
\centering
\begin{tabular}{|l|} \hline
obj\_type type; \\
union \{ \\
\multicolumn{1}{|l|}{\hspace{2em}graph\_t *g;} \\
\multicolumn{1}{|l|}{\hspace{2em}graph\_t *sg;} \\
\multicolumn{1}{|l|}{\hspace{2em}node\_t *n;} \\
\multicolumn{1}{|l|}{\hspace{2em}edge\_t *e;} \\
\} u; \\
gvcolor\_t pencolor; \\
gvcolor\_t fillcolor; \\
pen\_type pen;   \\
double penwidth; \\
char *url; \\
char *tailurl;  \\
char *headurl;  \\
char *tooltip; \\
char *tailtooltip; \\
char *headtooltip; \\
char *target; \\
char *tailtarget; \\
char *headtarget; \\
\hline
\end{tabular}
\caption{Some fields in {\tt obj\_state\_t}}
\label{fig:obj}
\end{figure*}

\begin{description}
\item[{\tt type} and {\tt u}]
The {\tt type} field indicates what kind of graph object is currently
being rendered. The possible values are 
{\tt ROOTGRAPH\_OBJTYPE}, {\tt CLUSTER\_OBJTYPE},
{\tt NODE\_OBJTYPE} and {\tt EDGE\_OBJTYPE},
indicating the root graph, a cluster subgraph, a node and an edge,
respectively.
A pointer to the actual object is available via the subfields
{\tt u.g}, {\tt u.sg}, {\tt u.n} and {\tt u.e}, respectively, 
of the union {\tt u}.
\item[{\tt pencolor}]
The {\tt gvcolor\_t} value indicating the color used to draw lines,
curves and text.
\item[{\tt pen}]
The style of pen to be used. The possible values are
{\tt PEN\_NONE}, {\tt PEN\_DOTTED},
{\tt PEN\_DASHED} and {\tt PEN\_SOLID}.
\item[{\tt penwidth}]
The size of the pen, in points.
Note that, by convention, a value of 0 indicates using the smallest
width supported by the output format.
\item[{\tt fillcolor}]
The {\tt gvcolor\_t} value indicating the color used to fill
closed regions.
\end{description}
Note that font information is delivered as part of the 
{\tt textpara\_t} value passed to the {\tt textpara} function.

As for the url, tooltip and target fields, these will point to
the associated attribute value of the current graph object,
assuming it is defined and that the renderer support map, tooltips,
and targets, respectively (cf. Section~\ref{sec:plugin}).

\subsection{Color information}
\label{sec:color}

There are five ways in which a color can be specified in \gviz:
RGB + alpha, HSV + alpha, CYMK, color index, and color name.
In addition, the RGB + alpha values can be stored as bytes, words
or doubles.

A color value in \gviz\ has the type {\tt gvcolor\_t}, containing
two fields: a union {\tt u}, containing the color data, and the
{\tt type} field, indicating which color representation is used in
the union. Table~\ref{table:color} describes the allowed color types,
and the associated union field.
\begin{table*}[htb]
\centering
\begin{tabular}[t]{|l|p{3.5in}|l|} \hline
\multicolumn{1}{|c|}{Type} & \multicolumn{1}{c|}{Description} & \multicolumn{1}{c|}{Field}\\ \hline
{\tt RGBA\_BYTE} & RGB + alpha format represented as 4 bytes from 0 to 255 & {\tt u.rgba} \\
{\tt RGBA\_WORD} & RGB + alpha format represented as 4 words from 0 to 65535 & {\tt u.rrggbbaa} \\
{\tt RGBA\_DOUBLE} & RGB + alpha format represented as 4 doubles from 0 to 1 & {\tt u.RGBA} \\
{\tt HSVA\_DOUBLE} & HSV + alpha format represented as 4 doubles from 0 to 1 & {\tt u.HSVA} \\
{\tt CYMK\_BYTE} & CYMK format represented as 4 bytes from 0 to 255 & {\tt u.cymk} \\
{\tt COLOR\_STRING} & text name & {\tt u.string} \\
{\tt COLOR\_INDEX} & integer index & {\tt u.index} \\
\hline
\end{tabular}
\caption{Color type representations}
\label{table:color}
\end{table*}

Before a color is used in rendering, \gviz\ will process a color description
provided by the input graph into a form desired by the renderer.
This is three step procedure. First, \gviz\ will see if the color matches
the renderer's known colors, if any.
If so, the color representation is {\tt COLOR\_STRING}. Otherwise, the
library will convert the input color description into the renderer's
preferred format. Finally, if the
renderer also provides a {\tt resolve\_color} function, \gviz\ will then
call that function, passing a pointer to the current color value. The
renderer then has the opportunity to adjust the value, or convert it
into another format. In a typical case, if a renderer uses a color map,
it may request RGB values as input, and then store an associated
color map index using the {\tt COLOR\_INDEX} format. If the renderer
does a conversion to another color type, it must 
reset the {\tt type} field to indicate this.
It is this last representation which will be passed to the renderer's
drawing routines.
The renderer's known colors and preferred color format are
described in Section~\ref{sec:plugin} below.

