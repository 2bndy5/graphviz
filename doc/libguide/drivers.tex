\section{The \gviz\ context}
\label{sec:gvc}

Up to now, we have used a \gviz\ context \gvc\ without
considering its purpose. As suggested earlier, this value is used
to store various layout information that is independent of a particular
graph and its attributes. 
It holds the data associated with plugins, parsed-command lines, 
script engines, and anything else with a scope potentially larger 
than one graph, up to the scope of the application.
In addition, it maintains lists of the
available layout algorithms and renderers; it also records the most recent
layout algorithm applied to a graph. It can be used to specify multiple
renderings of a given graph layout into different associated files.
It is also used to store various global information used during
rendering.

There should be just one \gvc\ created for the entire 
duration of an application.
A single \gvc\ value can be used with multiple graphs, though
with only one graph at a time. In addition, if {\tt gvLayout()}
was invoked for a graph and \gvc, then {\tt gvFreeLayout()} should
be called before using {\tt gvLayout()} again, even on the same graph.
 
An instance of a \gvc\ can be created by a call to
\begin{verbatim}
    extern GVC_t *gvNEWcontext(char **info, char *user);
\end{verbatim}
The first argument is an array of three character pointers
providing version information; see Section~\ref{sec:info} below for a
description of this data. The second argument is a string giving a
name for the user. If desired, the application can call the library
function {\tt gvUsername()} to obtain this value. These strings
are stored in the \gvc\ and used in various messages and comments.

For convenience, the \gviz\ library provides a simple way to 
create a context:
\begin{verbatim}
    extern GVC_t *gvContext();
\end{verbatim}
which is what we have used in the examples shown here.
This uses version information created when \gviz\ was built, plus
the value returned by {\tt gvUsername()}.

One can initialize a \gvc\ to record a list of graphs, layout algorithms
and renderers. To do this, the application should
call the function {\tt gvParseArgs}:
\begin{verbatim}
  extern void gvParseArgs(GVC_t* gvc, int argc, char* argv[]);
\end{verbatim}
This function takes the context value, plus an array of strings
using the same conventions as the parameters to {\tt main} function
in a C program. In particular, {\tt argc} should be the number of
values in {\tt argv}. If {\tt argv[0]} is the name of one of the
layout algorithms, this will be bound to the \gvc\ value and used
at layout time.
The remaining {\tt argv} values, if any, are interpreted exactly like
the allowed command line flags for any \gviz\ program.
Thus, {\tt "-T"} can be used to set the output type, and {\tt "-o"}
can be used to specify the output files.

For example, the application can use a synthetic argument list
\begin{verbatim}
    GVC_t* gvc = gcContext();
    char* args[] = {
        "dot",
        "-Tgif",       /* gif output */
        "-oabc.gif"    /* output to file abc.gif */
    };
    gvParseArgs (gvc, sizeof(args)/sizeof(char*), args);
\end{verbatim}
to specify a dot layout in {\tt GIF} output written to the file {\tt abc.gif}.
Another approach is to use a program's actual argument list, 
after removing flags not handled by \gviz.
 
Most of the information is stored in a \gvc\ value for use during
rendering. However,
if the {\tt argv} array contains non-flag arguments, i.e., strings
after the first not beginning with {\tt "-"}, these are taken
to be input files defining a stream of graphs to be drawn. 
These graphs can be accessed by calls to {\tt gvNextInputGraph}.

Once the \gvc\ has been initialized this way, the application can
call {\tt gvNextInputGraph} to get each input graph in sequence, and
then invoke {\tt gvLayoutJobs} and {\tt gvRenderJobs}
to do the specified layouts and renderings. See
Appendix~\ref{sec:dot} for a typical example of this approach.

We note that {\tt gvLayout} basically attaches the graph
and layout algorithm to the \gvc, as would be done by
{\tt gvParseArgs}, and then invokes {\tt gvLayoutJobs}. A similar
remark holds for {\tt gvRender} and {\tt gvRenderJobs}.
 
\subsection{Application-specific data}
\label{sec:info}

It is sometimes useful to supply version information.
For example, some renderers in \gviz\ the library version used
to create the output file. To do this, they rely on 
the application providing an array 
\begin{verbatim}
    extern char* Info[3];
\end{verbatim}
giving the desired version information.
The three strings should be the name of the application, the version
of the application, and a build date. For example, \dot\ might provide
\begin{verbatim}
    char *Info[] = {
        "dot",              /* Program */
        "1.8.10",           /* Version */
        "16 Dec 2006"       /* Build Date */
    };
\end{verbatim}

