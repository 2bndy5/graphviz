\section{A sample program: {\tt simple.c}}
\label{sec:simple}
This following code illustrates an application which uses \gviz\ to
position a graph using the \dot\ layout and then write the output
using the {\tt plain} format.
An application can replace the call to {\tt gvRender} with its own
function for rendering the graph, using the layout information
encoded in the graph structure (cf. Section~\ref{sec:layout_info}).
\begin{verbatim}
#include <gvc.h>

int main(int argc, char **argv)
{
    GVC_t *gvc; 
    Agraph_t *g;
    FILE *fp;

    gvc = gvContext();

    if (argc > 1)
        fp = fopen(argv[1], "r");
    else    
        fp = stdin;
    g = agread(fp, 0);

    gvLayout(gvc, g, "dot");

    gvRender(gvc, g, "plain", stdout);

    gvFreeLayout(gvc, g);

    agclose(g);

    return (gvFreeContext(gvc));
}
\end{verbatim}
\newpage

\section{A sample program: {\tt dot.c}}
\label{sec:dot}
This example shows how an application might read a stream of input graphs,
lay out each, and then use the \gviz\ renderers to write the drawings
to an output file. Indeed, this
is precisely how the \dot\ program is written, ignoring some signal
handling, its specific declaration of 
the {\tt Info} data (cf. Section~\ref{sec:info}), and a few other
minor details.
\begin{verbatim}
#include <gvc.h>
    
int main(int argc, char **argv)
{   
    Agraph_t *g, *prev = NULL;
    GVC_t *gvc;

    gvc = gvContext();
    gvParseArgs(gvc, argc, argv);
    
    while ((g = gvNextInputGraph(gvc))) {
        if (prev) {
            gvFreeLayout(gvc, prev);
            agclose(prev);
        }
        gvLayoutJobs(gvc, g); 
        gvRenderJobs(gvc, g);
        prev = g;
    }
    return (gvFreeContext(gvc));
}

\end{verbatim}
\newpage

\section{A sample program: {\tt demo.c}}
\label{sec:demo}
This example provides a modification of the previous example. Again it
relies on the \gviz\ renderers, but now it creates the graph dynamically
rather than reading the graph from a file.   
\begin{verbatim}
#include <gvc.h>

int main(int argc, char **argv)
{
    Agraph_t *g;
    Agnode_t *n, *m;
    Agedge_t *e;
    Agsym_t *a;
    GVC_t *gvc;

    /* set up a graphviz context */
    gvc = gvContext();

    /* parse command line args - minimally argv[0] sets layout engine */
    gvParseArgs(gvc, argc, argv);

    /* Create a simple digraph */
    g = agopen("g", Agdirected);
    n = agnode(g, "n", 1);
    m = agnode(g, "m", 1);
    e = agedge(g, n, m, 0, 1);

    /* Set an attribute - in this case one that affects the visible rendering */
    agsafeset(n, "color", "red", "");

    /* Compute a layout using layout engine from command line args */
    gvLayoutJobs(gvc, g);

    /* Write the graph according to -T and -o options */
    gvRenderJobs(gvc, g);

    /* Free layout data */
    gvFreeLayout(gvc, g);

    /* Free graph structures */
    agclose(g);

    /* close output file, free context, and return number of errors */
    return (gvFreeContext(gvc));
}
\end{verbatim}
